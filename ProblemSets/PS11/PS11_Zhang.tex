\documentclass{article}
\usepackage{graphicx} 
\usepackage[utf8]{inputenc}
\usepackage{csquotes}
\usepackage{natbib}
\usepackage{setspace}
\usepackage{parskip}
\usepackage[colorlinks=true, allcolors=blue]{hyperref}


\title{Predicting the Acceptance of Personal Loans}
\author{Haotian Zhang}
\date{March 2025}

\begin{document}
\doublespacing
\maketitle

\section{Introduction}
In today’s rapidly evolving and highly competitive financial environment, banks face mounting pressure to innovate their customer engagement strategies, diversify revenue streams, and enhance operational efficiency \citep{Krasnikov2009TheIndustry}. A critical challenge for many financial institutions is optimizing the composition of their customer portfolios—specifically, striking an effective balance between liability customers, who maintain deposits, and asset customers, who generate interest income through lending products \citep{Nikiel2002CUSTOMERBANKING}. Thera Bank, a U.S.-based, growth-oriented retail bank, epitomizes this challenge. Currently, the majority of its customers are liability holders with varying levels of deposits. While these customers contribute to funding stability, their direct impact on revenue is relatively limited when compared to asset customers, who are more directly linked to income-generating loan products \citep{Machauer2001SegmentationAttitudes}.

In response to this structural imbalance, Thera Bank has embarked on a strategic shift focused on expanding its asset customer base. Specifically, the bank is pursuing the conversion of existing liability customers into personal loan holders—an approach that not only leverages established customer relationships, thereby reducing acquisition costs, but also has the potential to significantly increase customer lifetime value \citep{Moro2014ATelemarketing}. Central to this strategy is the need to determine which liability customers are most likely to respond positively to personal loan offers. This question lies at the heart of the present research, which aims to help the bank identify high-potential target customers in order to more effectively tailor its marketing strategies and optimize promotional resource allocation. Accurately predicting these conversion probabilities is essential not only for optimizing marketing spend and improving response rates, but also for increasing the return on promotional investments \citep{Fenton2007VisualisingChange:}.

Initial results from Thera Bank’s targeted campaigns have been encouraging. Over the past year, the bank recorded a personal loan conversion rate exceeding 9\%, a figure that surpasses internal benchmarks and validates the promise of targeted marketing. Building on this momentum, the retail marketing team is now seeking more advanced methods to enhance campaign precision and scale outreach. The team contends that machine learning offers a robust and scalable framework for data-driven targeting—one capable of accurately predicting which customers are most likely to adopt a loan product \citep{Alaraj2021ModellingNetworks}.

To this end, the current study develops a predictive classification model that estimates the likelihood of liability customers converting to personal loan users, based on variables such as demographics, financial behavior, and previous engagement with the bank. By uncovering patterns that distinguish likely converters from non-responders, the model is designed to support more efficient customer segmentation and tailored marketing interventions. In addition to enhancing campaign performance, this approach also advances personalization efforts by aligning offers with customer needs and preferences.

Broadly speaking, this research contributes to the expanding literature on predictive analytics in financial services by demonstrating how data-driven modeling can inform customer segmentation and marketing strategy in real-world banking contexts. The findings are relevant not only to practitioners seeking operational insights but also to scholars exploring the applications of machine learning in consumer financial behavior, marketing analytics, and strategic decision-making.

\section{Literature Review}
In an era of increasing data availability and technological advancement, retail banks are undergoing a paradigm shift toward data-driven marketing and decision-making \citep{He2022ImpactStudy}. Predictive analytics has become central to understanding customer behavior, particularly in customer acquisition, product cross-selling, and loan adoption \citep{Boustani2024ImprovingNetworks}. One critical challenge lies in identifying which liability customers, who contribute deposit-based capital, are most likely to convert into asset customers through products such as personal loans \citep{Chang2024TowardsStudy}. Accurate prediction of such conversions can significantly enhance marketing efficiency, reduce acquisition costs, and increase customer lifetime value (CLV) \citep{Ulug2025OptimizedProgramming}.

The application of predictive analytics in retail banking has been widely studied in areas such as credit scoring, churn prediction, and loan default forecasting \citep{Singh2024InvestigatingManagement}. However, research focusing on loan product adoption from existing customers—especially liability to asset transition—is relatively nascent \citep{Vaduva2024ImprovingTechniques}. Traditional marketing practices, such as rule-based segmentation and demographic targeting, are increasingly giving way to machine learning techniques that leverage complex, non-linear customer behaviors \citep{deWaal2024ConsumersLearning}.

In the context of personal loan marketing, predictive modeling allows banks to create highly targeted campaigns by identifying customers with the highest likelihood of responding to offers \citep{Rahman2024TRANSFORMINGLEARNING}. As noted by \cite{BharathiS2022AnCustomers}, the integration of customer-level behavioral features (e.g., transaction patterns, credit card usage) has dramatically improved classification accuracy in banking environments. These developments underline the importance of choosing the right algorithm—not only for predictive performance, but also for interpretability, scalability, and deployability \citep{Al-Quraishi2025BridgingPrediction}.

In the following part of Literature review, I will show some literatures of pros and cons of logistic regression, tree model and knn methods and decide which is the better model for my project. 

\section{Data}
Including some descriptive statistics and some visualization pictures of the data set.  

\section{Methods}
1. Logistic regression 
2. Tree Model (optional)
3. knn methods (optional)

\section{Findings}
Use confusion matrix to evaluate the models

\section{Conclusion}



\bibliographystyle{apalike}  
\bibliography{references}
\end{document}
