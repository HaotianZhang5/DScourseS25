% Fonts/languages
\documentclass[12pt,english]{exam}
\IfFileExists{lmodern.sty}{\usepackage{lmodern}}{}
\usepackage[T1]{fontenc}
\usepackage[latin9]{inputenc}
\usepackage{babel}
\usepackage{mathpazo}
%\usepackage{mathptmx}
% Colors: see  http://www.math.umbc.edu/~rouben/beamer/quickstart-Z-H-25.html
\usepackage{color}
\usepackage[dvipsnames]{xcolor}
\definecolor{byublue}     {RGB}{0.  ,30. ,76. }
\definecolor{deepred}     {RGB}{190.,0.  ,0.  }
\definecolor{deeperred}   {RGB}{160.,0.  ,0.  }
\newcommand{\textblue}[1]{\textcolor{byublue}{#1}}
\newcommand{\textred}[1]{\textcolor{deeperred}{#1}}
% Layout
\usepackage{setspace} %singlespacing; onehalfspacing; doublespacing; setstretch{1.1}
\setstretch{1.2}
\usepackage[verbose,nomarginpar,margin=1in]{geometry} % Margins
\setlength{\headheight}{15pt} % Sufficent room for headers
\usepackage[bottom]{footmisc} % Forces footnotes on bottom
% Headers/Footers
\setlength{\headheight}{15pt}	
%\usepackage{fancyhdr}
%\pagestyle{fancy}
%\lhead{For-Profit Notes} \chead{} \rhead{\thepage}
%\lfoot{} \cfoot{} \rfoot{}
% Useful Packages
%\usepackage{bookmark} % For speedier bookmarks
\usepackage{amsthm}   % For detailed theorems
\usepackage{amssymb}  % For fancy math symbols
\usepackage{amsmath}  % For awesome equations/equation arrays
\usepackage{array}    % For tubular tables
\usepackage{longtable}% For long tables
\usepackage[flushleft]{threeparttable} % For three-part tables
\usepackage{multicol} % For multi-column cells
\usepackage{graphicx} % For shiny pictures
\usepackage{subfig}   % For sub-shiny pictures
\usepackage{enumerate}% For cusomtizable lists
\usepackage{listings} % For verbatim code
\usepackage{pstricks,pst-node,pst-tree,pst-plot} % For trees
% Bib
\usepackage[authoryear]{natbib} % Bibliography
\usepackage{url}                % Allows urls in bib
% TOC
\setcounter{tocdepth}{4}
% Links
\usepackage{hyperref}    % Always add hyperref (almost) last
\hypersetup{colorlinks,breaklinks,citecolor=black,filecolor=black,linkcolor=byublue,urlcolor=blue,pdfstartview={FitH}}
\usepackage[all]{hypcap} % Links point to top of image, builds on hyperref
\usepackage{breakurl}    % Allows urls to wrap, including hyperref
\pagestyle{head}
\firstpageheader{\textbf{\class\ - \term}}{\textbf{\examnum}}{\textbf{Due: May. 1\\ beginning of class}}
\runningheader{\textbf{\class\ - \term}}{\textbf{\examnum}}{\textbf{Due: May. 1\\ beginning of class}}
\runningheadrule
\newcommand{\class}{Econ 5253}
\newcommand{\term}{Spring 2025}
\newcommand{\examdate}{Due: May 1, 2025}
% \newcommand{\timelimit}{30 Minutes}
\noprintanswers                         % Uncomment for no solutions version
\newcommand{\examnum}{Problem Set 12}           % Uncomment for no solutions version
% \printanswers                           % Uncomment for solutions version
% \newcommand{\examnum}{Problem Set 12 - Solutions} % Uncomment for solutions version
\begin{document}
This problem set will give you practice with implementing some of the structural modeling techniques we discussed in class. A portion of the problem set will be closely related to Problem Set 7, which introduced methods for imputing missing data.
As with the previous problem sets, you will submit this problem set by pushing the document to \emph{your} (private) fork of the class repository. You will put this and all other problem sets in the path \texttt{/DScourseS25/ProblemSets/PS12/} and name the file \texttt{PS12\_LastName.*}. Your OSCER home directory and GitHub repository should be perfectly in sync, such that I should be able to find these materials by looking in either place. Your directory should contain at least three files:
\begin{itemize}
    \item \texttt{PS12\_LastName.R} (you can also do this in Python or Julia if you prefer)
    \item \texttt{PS12\_LastName.tex}
    \item \texttt{PS12\_LastName.pdf}
\end{itemize}
\begin{questions}
\question Type \texttt{git pull origin master} from your OSCER \texttt{DScourseS25} folder to make sure your OSCER folder is synchronized with your GitHub repository. 
\question Synchronize your fork with the class repository by doing a \texttt{git fetch upstream} and then merging the resulting branch. 
\question Install the following R packages (if you don't already have them installed):
\begin{itemize}
    \item \texttt{sampleSelection}
    \item \texttt{tidyverse}
    \item \texttt{modelsummary}
\end{itemize}
\question Using R or Python, load the file \texttt{wages12.csv} (located in the current directory) in as a data frame. This data set contains information on $\approx 2,250$ women who were working in the US in 1988. The variables should be self-explanatory. \texttt{exper} refers to how long (in years) each woman has worked at an employer, and \texttt{hgc}, which refers to how many years of schooling each woman has completed. \texttt{union} is a binary variable indicating whether or not the woman is currently holding a union job, and \texttt{kids} is an indicator for wither the woman has at least one children living at home.
\question Format the \texttt{college}, \texttt{married}, and \texttt{union} variables as factors.
\question Use \texttt{modelsummary} to produce a summary table of this data frame. Include it in your \LaTeX writeup and discuss whether the results make sense.
At what rate are log wages missing? Do you think the \texttt{logwage} variable is most likely to be MCAR, MAR, or MNAR?
\question As in PS7, perform the following imputation methods for missing \texttt{logwage} observations. \textbf{For each imputation method, estimate the following linear regression model:}
\[
    logwage_{i} = \beta_{0} +  \beta_{1}hgc_{i} + \beta_{2}union_{i} + \beta_{3}college_{i} + \beta_{4}exper_{i} + \beta_{5}exper^{2}_{i} + \eta_{i}
\]
Our coefficient of interest is $\beta_{1}$ which can be interpreted as the returns to schooling (where schooling is thought of as an ``investment'' in ``human captial''---we are curious what the Return On Investment is).
\begin{itemize}
    \item estimate the regression using only complete cases (i.e. do listwise deletion on the log wage variable ... this assumes log wages are Missing Completely At Random)
    \item perform mean imputation to fill in missing log wages
    \item use the \texttt{sampleSelection} package to correct for possible non-random missingness in the wages. To do this, follow these steps:
        \begin{itemize}
            \item create a new variable called \texttt{valid} which flags log wage observations that are not missing
            \item recode the log wage variable so that \emph{invalid} observations are now equal to 0
            \item estimate the Heckman selection (or ``Heckit'')  model by issuing the following code:
        \end{itemize}
\end{itemize}
\begin{footnotesize}
\begin{lstlisting}[language=R]
selection(selection = valid ~ hgc + union + college + exper + married + kids,
          outcome = logwage ~ hgc + union + college + exper + I(exper^2),
          data = wagedata, method = "2step")
\end{lstlisting}
\end{footnotesize}
Once you have finished all of this, use \texttt{modelsummary} to create one regression table which has the estimates of the first three regression models. Include this table in your .tex writeup. If \texttt{modelsummary} is not able to do this easily, you may use an AI product to create the table for you.
The true value of $\hat{\beta}_{1}=0.091$. Comment on the differences of $\hat{\beta}_{1}$ across the models. What patterns do you see? What can you conclude about the veracity of the various imputation methods?
\question Using the same data, estimate a probit model of preferences for working in a union job. To do this, use the \texttt{glm} function. The utility model has the following form:
\[
    u_{i} = \alpha_{0} +  \alpha_{1}hgc_{i} + \alpha_{2}college_{i} + \alpha_{3}exper_{i} + \alpha_{4}married_{i} + \alpha_{5}kids_{i} + \varepsilon_{i}
\]
where $u_{i}$ is the utility of working in a union job (relative to working in a non-union job).
\question Assess the impact of a counterfactual policy in which there is no preference or penalty to wives or mothers for working in union jobs. To do so, follow these steps: (if you don't remember how to do this, there is code in the GitHub repository in the file \texttt{example.R} in the \texttt{LectureNotes/25\_26-Discrete-Choice} folder)
\begin{itemize}
    \item compute predicted probabilities of the model
    \item change the coefficients on \texttt{married} and \texttt{kids} to equal zero
    \item compute predicted probabilities associated with the new parameter values
    \item compare the average of each set of predicted probabilities
\end{itemize}
Do you think that the model you estimated above is realistic? Why or why not?
\question Compile your .tex file, download the PDF and .tex file, and transfer it to your cloned repository on OSCER. There are many ways to do this;  you may ask an AI chatbot or simply drag-and-drop using VS Code. Do \textbf{not} put these files in your fork on your personal laptop; otherwise git will detect a merge conflict and that will be a painful process to resolve.
\question You should turn in the following files: .tex, .pdf, and any additional scripts (e.g. .R, .py, or .jl) required to reproduce your work.  Make sure that these files each have the correct naming convention (see top of this problem set for directions) and are located in the correct directory (i.e. \texttt{\textasciitilde/DScourseS25/ProblemSets/PS12}).
\question Synchronize your local git repository (in your OSCER home directory or on your local machine) with your GitHub fork by using the commands in Problem Set 2 (i.e. \texttt{git add}, \texttt{git commit -m ''message''}, and \texttt{git push origin master}). More simply, you may also just go to your fork on GitHub and click the button that says ``Fetch upstream.'' Then make sure to pull any changes to your local copy of the fork. Once you have done this, issue a \texttt{git pull} from the location of your other local git repository (e.g. on your personal computer). Verify that the PS12 files appear in the appropriate place in your other local repository.
\end{questions}
\end{document}
